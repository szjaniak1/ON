\documentclass{article}
\usepackage[T1]{fontenc}
\usepackage[utf8]{inputenc}
\usepackage[a4paper, total={6in, 11in}]{geometry}
\usepackage{algorithm}
\usepackage{amsfonts}
\usepackage{algpseudocode}
\usepackage{float}
\usepackage{graphicx}
\usepackage{subcaption}
\usepackage{amsmath}

\title{%
	Obliczenia naukowe \\
	\large Lista 5}
\author{Szymon Janiak}
\begin{document}
\maketitle

\section*{Opis problemu}
	Głównym problemem jest rozwiązanie równania liniowego\\
	\centerline{$Ax = b$},
	gdzie macierz A jest rzadką, tj. mającą dużą elementów zerowych, i blokową o następującej strukturze:\\
	\[
	A = \begin{bmatrix}
	A_1 & C_1 & 0 & 0 & \dots & 0 \\
	B_2 & A_2 & C_2 & 0 & \dots & 0 \\
	0 & B_3 & A_3 & C_3 & \dots & 0 \\
	\vdots & \vdots & \vdots & \ddots & \vdots \\
	0 & \dots & 0 & B_{v-2} & A_{v-2} & C_{v-2} \\
	0 & \dots & 0 & 0 & B_{v-1} & A_{v-1} \\
	0 & \dots & 0 & 0 & 0 & B_v A_v \\
	\end{bmatrix}
	\]
	i wektora prawych stron $\mathbf{b} \in \mathbb{R}^{n \times n}$, gdzie $n \geq 4$.\\
	\\
	Niech $v = \frac{n}{\ell}$, zakładając, że $n$ jest podzielne przez $\ell$, gdzie $\ell$ jest rozmiarem wszystkich kwadratowych macierzy wewnętrznych (bloków): $A_k$, $B_k$ i $C_k$. Mianowicie, $A_k \in \mathbb{R}^{\ell \times \ell}$, dla $k = 1, \ldots, v$ jest macierzą gęstą, $0$ jest kwadratową macierzą zerową stopnia $\ell$, a macierz $B_k \in \mathbb{R}^{\ell \times \ell}$, dla $k = 2, \ldots, v$ ma następującą postać:\\
	\[
	B_k = \begin{bmatrix}
	    0 & \dots & 0 & b_{k1} \\
	    0 & \dots & 0 & b_{k2} \\
	    \vdots & \vdots & \vdots & \vdots \\
	    0 & \dots & 0 & b_{k\ell}
	\end{bmatrix}
	\]\\

	Macierz $B_k$ ma tylko jedną, ostatnią, kolumnę niezerową. Natomiast $C_k \in \mathbb{R}^{\ell \times \ell}$, dla $k = 1, \ldots, v - 1$, jest macierzą diagonalną:\\
	\[
	C_k = \begin{bmatrix}
	    c_{k1} & 0 & 0 & \dots & 0 \\
	    0 & c_{k2} & 0 & \dots & 0 \\
	    \vdots & \vdots & \vdots & \ddots & \vdots \\
	    0 & \dots & 0 & c_{k(\ell-1)} & 0 \\
	    0 & \dots & 0 & 0 & c_{k\ell}
	\end{bmatrix}
	\]
	Dodatkowym wymaganiem związanym z problemem jest zadbanie o złożoność czasową i pamięciową rozwiązania ze względu na potrzebe obliczania macierzy o dużej wielkośći. Należy zapamietywać jedynie elementy niezerowe, gdyż nasz program będzie  pracował z macierzami rzadkimi oraz optymalizacji standardowych algorytmów w celu usprawnienia obliczeń.

\section*{Rozwiązanie}
\subsection*{Problem złożoności pamięciowej}
	Rozwiązanie korzysta z pakietu SparseArrays z biblioteki standardowej. Wykorzystana struktura danych o nazwie SparseMatrixSCS przechowuje jedynie niezerowe wartości macierzy co pozwala zaoszczędzić sporo pamięci w porównaniu ze zwykłej tablicy dwuwymiarowej która zajmowałaby aż $O(n^2)$ miejsca. W analizie naszego rozwiązania i wyników będziemy zakładać, że odczyt z tej struktury odbywa się w czasie stałym.

\subsection*{Użyte algorytmy i ich optymalizacje}
\subsubsection*{Eliminacja Gaussa}




\end{document}