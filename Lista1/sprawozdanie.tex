\documentclass{article}
\usepackage[T1]{fontenc}
\usepackage[a4paper, total={6in, 8in}]{geometry}
\usepackage{algorithm}
\usepackage{algpseudocode}
\title{%
	Obliczenia naukowe \\
	\large Lista 1}
\author{Szymon Janiak}
\begin{document}
\maketitle

\section{Zadanie 1}
\subsection{Macheps}
	$macheps$ to najmniejsza liczba $> 0$ taka, że $fl(1.0 + macheps) > 1.0$ i $fl(1.0 + macheps) = 1 + macheps$
\subsubsection{Opis problemu}
	Wyznaczenie iteracyjnie epsilonów maszynowych dla wszystkich dostępnych typów zmiennopozycyjnych Float16, Float32, Float64, zgodnych ze standardem IEEE 754.
\subsubsection{Rozwiązanie}
	\begin{algorithm}
	\caption{$macheps$ iteracyjnie}\label{alg:cap}
	\begin{algorithmic}
        \State $x \gets 1.0$
        \State $macheps \gets 1.0$
        \For{ $i$ to $50$ }
            \If{ $x + \frac{macheps}{2} = x$ }
            	\State break
            \EndIf
            \State $macheps \gets \frac{macheps}{2}$
        \EndFor
        \State \textbf{return} $macheps$
    \end{algorithmic}
    \end{algorithm}
\subsubsection{Wyniki}
	\begin{center}
        \begin{tabular}{|c||c|c|c|}
        \hline
            Źródło & Float16 & Float32 & Float64 \\
            \hline\hline
            Mój algorytm & 0.000977 & 1.1920929e-7 & 2.220446049250313e-16\\
            \hline
            eps() & 0.000977 & 1.1920929e-7 & 2.220446049250313e-16\\
            \hline
            float.h & - & 1.19209e-07 & 2.22045e-16 \\
        \hline
        \end{tabular}
    \end{center}
\subsubsection{Wnioski}
    Wraz z wzrostem precyzji zmniejsza sie $macheps$.

\subsection{Eta}
	$eta$ to najmniejsza liczba taka, że $eta > 0.0$
\subsubsection{Opis problemu}
    Wyznaczenie iteracyjnie liczb maszynowych eta dla wszystkich dostępnych typów zmiennopozycyjnych Float16, Float32, Float64, zgodnych ze standardem IEEE 754.
\subsubsection{Rozwiązanie}
    \begin{algorithm}
    \caption{$eta$ iteracyjnie}\label{alg:cap}
    \begin{algorithmic}
        \State $eta \gets 1.0$
        \For{ $i$ to $1100$ }
            \If{ $\frac{eta}{2} = 0.0$ }
            	\State break
            \EndIf
            \State $eta \gets \frac{eta}{2}$
        \EndFor
        \State \textbf{return} $eta$
    \end{algorithmic}
    \end{algorithm}
\subsubsection{Wyniki}
    \begin{center}
        \begin{tabular}{|c||c|c|c|}
        \hline
            Źródło & Float16 & Float32 & Float64\\
            \hline\hline
            Mój algorytm & 6.0e-8 & 1.0e-45 & 5.0e-324\\
            \hline
            nextfloat(0.0) & 6.0e-8 & 1.0e-45 & 5.0e-324\\
        \hline
        \end{tabular}
    \end{center}
\subsubsection{Wnioski}
    $MIN_{sub} = 2^{1-t}*2^{c_{min}}$\\
    gdzie $t$ to liczba cyfr mantysy, a $c_{min}=-2^{d-1}+2$ gdzie, $d$ oznacza liczbę bitów przeznaczonych na zapis cechy.
        \begin{center}
        \begin{tabular}{|c|c|c|c|}
        \hline
             & Float16 & Float32 & Float64 \\
            \hline
            $MIN_{sub}$ & 6.0e-8 & 1.0e-45 & 5.0e-324\\
        \hline
        \end{tabular}
    \end{center}
    Z tego wynika, że $eta = MIN_{sub}$

\subsection{$MIN_{nor}$}
    $MIN_{nor} = 2^{c_{min}}$\\
    gdzie $c_{min}$ jest tym samym co przy $MIN_{sub}$ 
        \begin{center}
        \begin{tabular}{|c|c|c|c|}
        \hline
             & Float16 & Float32 & Float64 \\
            \hline
            $MIN_{nor}$ & 6.104e-5 & 1.1754944e-38 & 2.2250738585072014e-308\\
            \hline
            $floatmin()$ & 6.104e-5 & 1.1754944e-38 & 2.2250738585072014e-308\\
        \hline
        \end{tabular}
    \end{center}
\subsubsection{Wnioski}
     $floatmin()$ dla danej arytmetyki jest równy z jej $MIN_{nor}$

\subsection{MAX}
\subsubsection{Opis Problemu}
    Wyznaczenie iteracyjnie liczbę MAX dla wszystkich typów zmiennoprzecinkowych, zgodnych ze standardem IEEE 754.
\subsubsection{Rozwiązanie}
    \begin{algorithm}
    \caption{$MAX$ iteracyjnie}\label{alg:cap}
    \begin{algorithmic}
        \State $max \gets prevfloat(1.0$
        \While{ $max * 2 \neq \infty $} 
            \State $max \gets MAX * 2$
        \EndWhile
        \State \textbf{return} $MAX$
    \end{algorithmic}
    \end{algorithm}
\subsubsection{Wyniki}
    \begin{center}
        \begin{tabular}{|c||c|c|c|}
        \hline
            Źródło & Float16 & Float32 & Float64\\
            \hline\hline
            Mój algorytm & 6.55e4 & 3.4028235e38 & 1.7976931348623157e308\\
            \hline
            floatmax() & 6.55e4 & 3.4028235e38 & 1.7976931348623157e308\\
            \hline
            float.h & - & 3.40282e+38 & 1.79769e+308\\
        \hline
        \end{tabular}
    \end{center}

\section{Zadanie 2}
\subsection{Twierdzenie Khan'a}
	Epislon maszynowy $macheps$ można otrzymać obliczając wyrażenie $3.0*(\frac{4.0}{3.0} - 1.0) - 1.0$
\subsection{Opis problemu}
    Sprawdzenie czy stwierdzenie Khan'a jest słuszne dla wszystkich typów zmiennopozycyjnych Float16, Float32, Float64.
\subsection{Wyniki}
    \begin{center}
        \begin{tabular}{|c||c|c|c|}
        \hline
            Źródło & Float16 & Float32 & Float64 \\
            \hline\hline
            Khan & -0.000977 & 1.1920929e-7 & -2.220446049250313e-16\\
            \hline
            eps() & 0.000977 & 1.1920929e-7 & 2.220446049250313e-16\\
        \hline
        \end{tabular}
    \end{center}
\subsection{Wnioski}
    Wyniki z wyrażenia od Khan'a różnią się znakiem dla Float16 i Float64.

\section{Zadanie 3}
\subsection{Opis problemu}
	Sprawdzenie rozmieszczenia liczb zmiennoprzecinkowych Float64 w standardzie IEEE 754 w przedziałach [1, 2], [$\frac{1}{2}$, 1] oraz [2, 4].
\subsection{Rozwiązanie}
	$\delta$ - krok o który będziemy powiększać początkowe liczby.\\
    Obserwując otrzymywane wyniki, będziemy mogli dostować wielkośc kroku, aby uzyskać zwiększenie o jeden bit na danym przedziale.
\subsection{Wyniki}
	$\delta=2^{-52}$ dla przedziału [1, 2]: 
	Różnica po jednym kroku\\
	0011111111110000000000000000000000000000000000001010110011000001 -- 1.00000000000982\\
	0011111111110000000000000000000000000000000000001010110011000010 -- 1.0000000000098201\\
	$\delta=\frac{2^{-52}}{2}$ dla przedziału [$\frac{1}{2}$, 1]:
	Tutaj musieliśmy podzielić krok przez 2 aby uzyskac taki sam rezultat jak wcześniej 
	Różnica po jednym kroku\\
	0011111111100000000000000000000000000000000000001001001101001100 -- 0.5000000000041864\\
	0011111111100000000000000000000000000000000000001001001101001101 -- 0.5000000000041865\\
	$\delta=2^{-52} * 2$ dla przedziału [2, 4]:
	Tutaj sytuacja odwrotna, długość kroku musiała zostać zwiększona
	Różnica po jednym kroku\\
	0100000000000000000000000000000000000000000000000111110111011101 -- 2.000000000014309\\
	0100000000000000000000000000000000000000000000000111110111011110 -- 2.0000000000143094\\
\subsection{Wnioski}
	W przedziale [1, 2] liczby występują co $\delta=2^{-52}$.\\
	W przedziale [$\frac{1}{2}$, 1] 2 razy częsciej, a w przedziale [2, 4] 2 razy rzadziej.

\section{Zadanie 4}
\subsection{Opis problemu}
    Znalezienie liczby w arytmetyce Float64 zgodnej ze standardem IEEE 754 liczbe zmiennopozycyjna $x$ w przedziale $1<x<2$ taką, że $x*(1/x) \neq 1$.
\subsection{Rozwiązanie}
    \begin{algorithm}
    \caption{find}\label{alg:cap}
    \begin{algorithmic}
        \State $delta \gets 2^{-52}$
        \For{ $k$ in $2^{-52} - 1$}
        	\State $x \gets 1 + k * \delta$
            \If{$(x * \frac{1}{x}) \neq 1 $}
                \State \textbf{return} $x$
            \EndIf
        \EndFor
    \end{algorithmic}
    \end{algorithm}
\subsection{Wyniki}
	Najmniejsza znaleziona wartość:
	1.000000057228997

\section{Zadanie 5}
\subsection{Opis problemu}
    Obliczenie iloczynu skalarnego dwóch wektorów:\\
    x = [2.718281828, -3.141592654, 1.414213562, 0.5772156649, 0.3010299957]\\
    y = [1486.2497, 878366.9879, -22.37492, 4773714.647, 0.000185049]\\
    używając pojedyńczej i podwójnej precyzji
\subsection{Rozwiązanie}
    \begin{enumerate}
        \item "w przód" t.j. $\sum^n_{i=1} x_i y_i$
        \item "w tył" t.j. $\sum^1_{i=n} x_i y_i$
        \item "liczby dodatnie od najwiekszego do najmniejszego a ujemne na odwrót"
        \item "liczby ujemne od najwiekszego do najmniejszego a dodatnie na odwrót"
    \end{enumerate}
\subsection{Wyniki}
    \begin{center}
        \begin{tabular}{|c||c|c|c|}
        \hline
            Źródło & Float32 & Float64 & Prawidłowy wynik \\
            \hline\hline
            "1" & -0.4999443 & 1.0251881368296672e-10 & -1.00657107000000e-11\\
             \hline
             "2" & -0.4543457 & -1.5643308870494366e-10 & -1.00657107000000e-11\\
             \hline
             "3"& -0.5 & 0.0 & -1.00657107000000e-11\\
             \hline
             "4"& -0.5 & 0.0 & -1.00657107000000e-11\\
        \hline
        \end{tabular}
    \end{center}

\section{Zadanie 6}
\subsection{Opis problemu}
    Policzenie wartości w arytmetyce Float64 następujących funkcji:\\
    \begin{center}
        $f(x)=\sqrt{x^2 + 1} - 1$\\ $g(x)=\frac{x^2}{\sqrt{x^2 +1}+1}$
    \end{center}
\subsection{Wyniki}
        \begin{center}
        \begin{tabular}{|c||c|c|}
        \hline
            $x$ & $f(x)$ & $g(x)$ \\
            \hline\hline
            $8^{-1}$ & -0.0077822185373186414 & 0.0077822185373187065 \\
            \hline
            $8^{-2}$ & 0.00012206286282867573 & 0.00012206286282875901\\
            \hline
            $8^{-3}$ & 1.9073468138230965e-6 & 1.907346813826566e-6\\
            \hline
            $8^{-4}$ & 2.9802321943606103e-8 & 2.9802321943606116e-8\\
            \hline
            $8^{-5}$ & 4.656612873077393e-10 & 4.6566128719931904e-10\\
            \hline
            $8^{-6}$ & 7.275957614183426e-12 & 7.275957614156956e-12\\
            \hline
            $8^{-7}$ & 1.1368683772161603e-13 & 1.1368683772160957e-13\\
            \hline
            $8^{-8}$ & 1.7763568394002505e-15 & 1.7763568394002489e-15\\
            \hline
            $8^{-9}$ & 0.0 & 2.7755575615628914e-17\\
            \hline
            $8^{-10}$ & 0.0 & 4.336808689942018e-19\\
        \hline
        \end{tabular}
    \end{center}
\subsection{Wnioski}
    Funkcja $f(x)$ zwróciła jeden ujemny wynik co nie jest możliwe, do tego od pewnego momentu zwraca wynik 0.0 co jest mało dokładne, stąd wniosek, że funkcja $g(x)$ wydaje się być bardziej wiarygodna.

\section{Zadanie 7}
\subsection{Opis problemu}
    Przybliżoną wartość pochodnej $f(x)$ w punkcie $x$ można obliczyć za pomocą następującego wzoru:\\
    \begin{center}
        $f^{'}(x_0)\approx\tilde{f^{'}}(x_0) = \frac{f(x_0+h) - f(x_0)}{h}$
    \end{center}
    Korzystając z tego wzoru do obliczania w arytmetyce Float64 przybliżonej wartości pochodnej funckji $f(x)=sinx+cos3x$ w punkcie $x_0 = 1$ oraz obliczamy błędy $|f^{'}(x_0) - \tilde{f^{'}}(x_0)|$ dla $h=2^{-n}$ (n = 0, 1, 2, ..., 54)
\subsection{Wyniki}
	\begin{center}
        \begin{tabular}{|c||c|c|}
        \hline
            $h$ & aproksymacja & różnica od pochodnej\\
            \hline\hline
            $2^{-3}$ & 0.6232412792975817 & 0.5062989976090435\\
            \hline
            $2^{-7}$ & 0.1484913953710958 & 0.03154911368255764\\
            \hline
            $2^{-12}$ & 0.11792723373901026 & 0.0009849520504721099\\
            \hline
            $2^{-31}$ & 0.11694216728210449 & 1.1440643366000813e-7\\
            \hline
            $2^{-32}$ & 0.11694192886352539 & 3.5282501276157063e-7\\
        \hline
        \end{tabular}
    \end{center}
\subsection{Wnioski}
    W wynikach możemy zaobserować ze od pewnego momentu zmniejszanie wartości $h$ wcale nie poprawia przybliżenia wartości pochodnej. Dzieje sie tak prawdopodbnie przez coraz to mniejsze wartości co może powodować problemy w arytmetyce zmiennoprzecinkowej.

\end{document}