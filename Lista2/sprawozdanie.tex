\documentclass{article}
\usepackage[T1]{fontenc}
\usepackage[a4paper, total={6in, 8in}]{geometry}
\usepackage{algorithm}
\usepackage{algpseudocode}

\title{%
	Obliczenia naukowe \\
	\large Lista 2}
\author{Szymon Janiak}
\begin{document}
\maketitle

\section{Zadanie 1}
\subsection{Opis problemu}
    Obliczenie iloczynu skalarnego róznymi funkcjami dwóch wektorów i porównanie wyników przy lekkiej zmianie danych wejściowych
\subsection{Rozwiązanie}
    \begin{enumerate}
        \item "w przód" t.j. $\sum^n_{i=1} x_i y_i$
        \item "w tył" t.j. $\sum^1_{i=n} x_i y_i$
        \item "liczby dodatnie od najwiekszego do najmniejszego a ujemne na odwrót"
        \item "liczby ujemne od najwiekszego do najmniejszego a dodatnie na odwrót"
    \end{enumerate}
\subsection{Wyniki}
    \begin{center}
        \begin{tabular}{|c|c|c|c|}
        \hline
            & Float64 stare dane & Float64 nowe dane & Prawidłowy wynik \\
            \hline\hline
            "1" & 1.0251881368296672e-10 & -0.004296342739891585 & -1.00657107000000e-11\\
             \hline
             "2" & -1.5643308870494366e-10 & -0.004296342998713953 & -1.00657107000000e-11\\
             \hline
             "3" & 0.0 & -0.004296342842280865 & -0.004296342842280865 \\
             \hline
             "4" & 0.0 & -0.004296342842280865 & -0.004296342842280865 \\
        \hline
        \end{tabular}
    \end{center}
\subsection{Wnioski}
	Przy usunięciu ostatniej 9 z $x_4$ oraz ostatniej 7 z $x_5$ dostajemy różne wyniki dla podwójnej precyzji.
	Po tak lekkiej zmianie danych możemy zauważyć, że wyniki dla wszystkich funkcji są znacznie bardziej przybliżone, prawie identyczne.
	Dla Float32 nie ma żadnej różnicy, gdyż jest to za mała precyzja. Według definicji jest to źle uwarunkowane zadanie.

\section{Zadanie 2}

\end{document}